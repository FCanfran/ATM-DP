\newpage
\section{Related Work}
\label{sec:related_work}

Recently, in Rost et al. \cite{rost2024seraph} authors formalize an extension to the query language Cypher that allows for evaluating  continuous queries over property graph streams according to a specified frequency, during a defined time interval. 
The main differences of their approach with our proposal are: first, the data model because they merge incoming PGs to the persistent database and, second, we evaluate continuous queries as new edges are added to the volatile part of the property graph (i.e.  as the PG evolves) by processing (identifying) patterns (subgraphs).
To our knowledge there is no other work approaching the problem of modeling and implementing a continuous query engine for detecting anomalous ATM transactions. In effect, according to the survey of Rahman et al. \cite{rahman2019comparative} most of the ATM fraud detection and prevention mechanisms are based on analyzing the spending behavior of cardholders in order to detect suspicious behavior. Furthermore, these authors claim that all these mechanisms are based on artificial intelligence, data mining, neural networks, Bayesian networks, artificial immunology systems, support vector machines, decision trees, machine learning, etc. The most important criteria to be consider for assessing these kind of methods are their accuracy, speed and cost. Accuracy refers to the capacity of effectively detecting a fraud, i.e. is a metric that measures how often a machine learning model correctly predicts the outcome. It can be computed by dividing the number of correct predictions by the total number of predictions. In Rahman's survey \cite{rahman2019comparative}, the relevant result after comparing different methods is that the accuracy is low in most of them. In addition, these methods need training processes and hence, big volume of data. In the same line is the work of Ahmed et al. \cite{ahmed2016survey} in which, again, the importance of detecting frauds in financial domains and  some methods based on artificial intelligence techniques are analyzed. To be concrete, authors study methods based on clustering  anomaly detection techniques. As we do, authors complain about the lack of real world data, one of the point we tackle in this work. 
In \cite{kian2022detection} authors present a system to block fraudulently issued transactions based on a big data clustering method to timely identify  abnormal transactions. They claim that by clustering credit card data and its transactions, it is possible to identify frequent and expensive purchases, and then to investigate the possibility of a crime to discover specific cases. In the conclusions of this work the need of identifying fraud patterns in the use of bank cards is highlighted. However, it is not proposed a specific way to model and deal with these patterns. Another work \cite{heryadi2016recognizing} provides a fraud recognition models based on debit card transaction dataset
from Indonesian bank. The starting point of this research is that ``fraudulent transaction contains ‘anomaly’ from the pattern of non-fraudulent transactions". Authors use classification models to detect anomalous patterns, obtaining an accuracy around 0.7 in both cases. Although there is a lot of work tackling the problem of detecting anomalous ATM transactions, most of them based on artificial intelligence techniques, there is not a clear characterization of the problem and what to solve it means or implies. As said before, we focus on providing an explicit characterization of what anomalous transaction  patterns are and a method  based on graph recognition and stream processing  computational model to detect them, with 100\% of accuracy.
