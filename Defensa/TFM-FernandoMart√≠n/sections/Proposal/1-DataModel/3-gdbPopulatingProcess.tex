\subsection{Graph database populating process}

\begin{comment}
Puntos a comentar:
- Neo4j Desktop y version - instalación y detalles. Cómo conectarse...

\end{comment}

\subsubsection{Neo4j graph database}
% Neo4j Desktop y version - instalación y detalles. Cómo conectarse...
Version: Neo4j 5.21.0 Community edition. 
\begin{itemize}
    \item Accessing it: by default it runs on localhost port 7474: \texttt{http://localhost:7474}.
    Start the neo4j service locally by: \texttt{sudo systemctl start neo4j}\\
    It can be also be accessed by the internal utility \texttt{cypher-shell}. Username: \texttt{neo4j} and password: \texttt{bisaurin}.
\end{itemize}

% TODO: Explain how to install / connect... all the details that are in the neo4.md file

\subsubsection{Neo4J PG creation - CSV to PG}
\begin{comment}
    - Creación de la GDB - de CSV a Neo4j PG: 
    - Poner primero los comandos cypher por separado, con las constraints de uniqueness
    y luego lo de poblar. Luego ya referir a que todo ello se tiene en un único script que 
    permite la ejecución directa (en lugar de paso a paso) en golang.
    - Incluir detalles específicos de cada CSV...
\end{comment}
Once all the CSV data tables representing all the nodes and relations were created, the process to create the PG in Neo4j is described in what follows. It is done used the cypher clauses to load CSV into Neo4j (see \href{https://neo4j.com/docs/cypher-manual/5/clauses/load-csv/}{\textit{load-csv cypher manual}}). 
However, for simplicity the \texttt{populatemodule} golang module that groups all the cypher commands needed to do the population of the Neo4j GDB was elaborated to do the process with a single command run.
Some notes on the usage:
\begin{itemize}
    \item Define the corresponding Neo4j URI, username and password in the \texttt{.env} file.
    % TODO: Esto es en mi local. Se recomienda por razones de seguridad... Sin embargo se puede configurar para que no sea % así. - En el cluster veremos. De momento no pongo nada de esto!
    % First, the CSV files were needed to be placed under the \texttt{/var/lib/neo4j/import} directory.
    \item All the CSVs (\texttt{atm.csv}, \texttt{bank.csv}, \texttt{card.csv}, \texttt{atm-bank-internal.csv}, \texttt{atm-bank-external.csv} 
    and \texttt{card-bank.csv}) need to be placed under the \texttt{/var/lib/neo4j/import} directory.
\end{itemize}

\paragraph{Process description:}

Note that first, prior to the population of the GDB, a uniqueness constraint on the IDs of each of the three different kind of nodes are added. This way we avoid having duplicated nodes with the same ID in the database. Therefore, when adding a new ATM node that has the same ID as another ATM already existing in the database, we are aware of this and we do not let this insertion to happen.

ID uniqueness constraints were created for the unique identifiers of all kind of nodes: Bank (\texttt{code}), ATM \texttt{ATM\_id} and Card (\texttt{number\_id}) with the following cypher directives:

\begin{center}
\lstset{style=cypherStyle}
\begin{lstlisting}[caption={Uniqueness ID constraints}]
            CREATE CONSTRAINT ATM_id IF NOT EXISTS
            FOR (a:ATM) REQUIRE a.ATM_id IS UNIQUE
    
            CREATE CONSTRAINT number_id IF NOT EXISTS
            FOR (c:Card) REQUIRE c.number_id IS UNIQUE
    
            CREATE CONSTRAINT code IF NOT EXISTS
            FOR (b:Bank) REQUIRE b.code IS UNIQUE
\end{lstlisting}
\end{center}

Then the different CSV files containing all the data tables of our data set, were loaded into the GDB with the following cypher directives.

% TODO: Esto es en mi local. Se recomienda por razones de seguridad... Sin embargo se puede configurar para que no sea
% así. - En el cluster veremos. De momento no pongo nada de esto!
% First, the CSV files were needed to be placed under the \texttt{/var/lib/neo4j/import} directory.

\paragraph{ATM (atm.csv)}

\begin{center}
\lstset{style=cypherStyle}
\begin{lstlisting}[caption={atm.csv}]
    LOAD CSV WITH HEADERS FROM 'file:///csv/atm.csv' AS row
    MERGE (a:ATM {
        ATM_id: row.ATM_id,
        loc_latitude: toFloat(row.loc_latitude),
        loc_longitude: toFloat(row.loc_longitude),
        city: row.city,
        country: row.country
    });
\end{lstlisting}
\end{center}

Some remarks:
\begin{itemize}
    \item \texttt{ATM} is the node label, the rest are the properties of this kind of node.
    \item Latitude and longitude are stored as float values; note that they could also be stored
    as cypher \textit{Point} data type. However for the moment it is left like this. In the future
    it could be converted when querying or directly be set as cypher point data type as property.
\end{itemize}

\paragraph{Bank (bank.csv)}

\begin{center}
\lstset{style=cypherStyle}
\begin{lstlisting}[caption={bank.csv}]
    LOAD CSV WITH HEADERS FROM 'file:///csv/bank.csv' AS row
    MERGE (b:Bank {
        name: row.name, 
        code: row.code, 
        loc_latitude: toFloat(row.loc_latitude), 
        loc_longitude: toFloat(row.loc_longitude)
    });
\end{lstlisting}
\end{center}

Note that the \texttt{code} is stored as a string and not as an integer, since to make it more clear it 
was already generated as a string code name.

\paragraph{ATM-Bank relationships (atm-bank-internal.csv and atm-bank-external.csv)}

\begin{center}
\lstset{style=cypherStyle}
\begin{lstlisting}[caption={atm-bank-internal.csv}]
    LOAD CSV WITH HEADERS FROM 'file:///csv/atm-bank-internal.csv' AS row
    MATCH (a:ATM {ATM_id: row.ATM_id})
    MATCH (b:Bank {code: row.code})
    MERGE (a)-[r:BELONGS_TO]->(b);
\end{lstlisting}
\end{center}

\begin{center}
\lstset{style=cypherStyle}
\begin{lstlisting}[caption={atm-bank-external.csv}]
    LOAD CSV WITH HEADERS FROM 'file:///csv/atm-bank-external.csv' AS row
    MATCH (a:ATM {ATM_id: row.ATM_id})
    MATCH (b:Bank {code: row.code})
    MERGE (a)-[r:INTERBANK]->(b);
\end{lstlisting}
\end{center}

\paragraph{Card (card.csv)}

\begin{center}
\lstset{style=cypherStyle}
\begin{lstlisting}[caption={card.csv}]
    LOAD CSV WITH HEADERS FROM 'file:///csv/card.csv' AS row
    MERGE (c:Card {
        number_id: row.number_id, 
        client_id: row.client_id, 
        expiration: date(row.expiration), 
        CVC: toInteger(row.CVC), 
        extract_limit: toFloat(row.extract_limit), 
        loc_latitude: toFloat(row.loc_latitude), 
        loc_longitude: toFloat(row.loc_longitude)});
\end{lstlisting}
\end{center}

Notes:
\begin{itemize}
    \item We do not include the fields that were generated to define the behavior of the card. They are only used for the generation of the transactions.
    \item \texttt{expiration}: set as \textit{date} data type.
\end{itemize}

\paragraph{Card-Bank relationships (card-bank.csv)}

\begin{center}
\lstset{style=cypherStyle}
\begin{lstlisting}[caption={card-bank.csv}]
    LOAD CSV WITH HEADERS FROM 'file:///csv/card-bank.csv' AS row
    MATCH (c:Card {number_id: row.number_id})
    MATCH (b:Bank {code: row.code})
    MERGE (c)-[r:ISSUED_BY]->(b);
\end{lstlisting}
\end{center}

%\subsubsection*{Dataset extensions}

\begin{comment}
- Interesting reference for coordinates and distances in cypher - https://lyonwj.com/blog/spatial-cypher-cheat-sheet 
- Transactions dataset Generation  - a transaction dataset simulator → useful with the description of fraud scenarios, that can be generated among the transactions dataset, also with customers and terminals info generation (similar to ATMs concept) 
\end{comment}

\textcolor{red}{$\rightarrow$ TODO: Describe the other population method}

\textcolor{green}{\rule{\linewidth}{0.4mm}}
