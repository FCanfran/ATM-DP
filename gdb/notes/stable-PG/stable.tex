\documentclass{article}
\usepackage[utf8]{inputenc}
\usepackage{graphicx} % Required for inserting images
\usepackage{hyperref}
\usepackage{subcaption}
\usepackage{float}
\usepackage{tcolorbox}
\usepackage{amsmath}
\usepackage{amssymb}
\usepackage{listings}% http://ctan.org/pkg/listings}
\usepackage{algorithm}
\usepackage{algorithmic}
\usepackage[toc,page]{appendix}
\usepackage[backend=biber]{biblatex}
\usepackage{multicol}
\usepackage{siunitx}
\usepackage{comment}
\usepackage{xcolor}
\usepackage{caption}
\usepackage{tikz}
\usetikzlibrary{shapes.geometric, arrows}

\tikzstyle{startstop} = [rectangle, rounded corners, 
minimum width=3cm, 
minimum height=1cm,
text centered, 
draw=black, 
fill=red!30]

\tikzstyle{io} = [trapezium, 
trapezium stretches=true, % A later addition
trapezium left angle=70, 
trapezium right angle=110, 
minimum width=3cm, 
minimum height=1cm, text centered, 
draw=black, fill=blue!30]

\tikzstyle{process} = [rectangle, 
minimum width=3cm, 
minimum height=1cm, 
text centered, 
text width=3cm, 
draw=black, 
fill=orange!30]

\tikzstyle{decision} = [diamond, 
minimum width=3cm, 
minimum height=1cm, 
text centered, 
draw=black, 
fill=green!30]
\tikzstyle{arrow} = [thick,->,>=stealth]

% To delete lstlisting caption "Listing x"
%\captionsetup[lstlisting]{labelformat=empty}

\lstdefinestyle{myStyle}{
    belowcaptionskip=1\baselineskip,
    breaklines=true,
    frame=none,
    numbers=none, 
    basicstyle=\footnotesize\ttfamily,
    keywordstyle=\bfseries\color{green!40!black},
    commentstyle=\itshape\color{purple!40!black},
    identifierstyle=\color{black},
    backgroundcolor=\color{white},
}

\lstdefinestyle{cypherStyle}{
    backgroundcolor=\color{white},   % choose the background color
    basicstyle=\footnotesize\ttfamily,        % the size of the fonts that are used for the code
    commentstyle=\itshape\color{purple!40!black},
    keywordstyle=\bfseries\color{green!40!black},
    breakatwhitespace=false,         % sets if automatic breaks should only happen at whitespace
    breaklines=true,                 % sets automatic line breaking
    captionpos=b,                    % sets the caption-position to bottom
    commentstyle=\color{gray},    % comment style
    deletekeywords={},            % if you want to delete keywords from the given language
    escapeinside={\%*}{*)},          % if you want to add LaTeX within your code
    extendedchars=true,              % lets you use non-ASCII characters; for 8-bits encodings only, does not work with UTF-8
    %firstnumber=1000,                % start line enumeration with line 1000
    frame=none,                    % adds a frame around the code
    keepspaces=true,                 % keeps spaces in text, useful for keeping indentation of code (possibly needs columns=flexible)
    language=SQL,                    % the language of the code
    morekeywords={*,IF, REQUIRE, FOR, IS, LOAD, CSV, WITH, HEADERS, MERGE, toFloat, toInteger, date},            % if you want to add more keywords to the set
    numbers=none,                    % where to put the line-numbers; possible values are (none, left, right)
    numbersep=5pt,                   % how far the line-numbers are from the code
    numberstyle=\tiny\color{mygray}, % the style that is used for the line-numbers
    rulecolor=\color{black},         % if not set, the frame-color may be changed on line-breaks within not-black text (e.g. comments (green here))
    showspaces=false,                % show spaces everywhere adding particular underscores; it overrides 'showstringspaces'
    showstringspaces=false,          % underline spaces within strings only
    showtabs=false,                  % show tabs within strings adding particular underscores
    stepnumber=1,                    % the step between two line-numbers. If it's 1, each line will be numbered
    stringstyle=\ttfamily,     % string literal style
    tabsize=2,                       % sets default tabsize to 2 spaces
}

%% Golang definition for listings
%% http://github.io/julienc91/lstlistings-golang
%%
\lstdefinelanguage{Golang}%
  {morekeywords=[1]{package,import,func,type,struct,return,defer,panic,%
     recover,select,var,const,iota,},%
   morekeywords=[2]{string,uint,uint8,uint16,uint32,uint64,int,int8,int16,%
     int32,int64,bool,float32,float64,complex64,complex128,byte,rune,uintptr,%
     error,interface},%
   morekeywords=[3]{map,slice,make,new,nil,len,cap,copy,close,true,false,%
     delete,append,real,imag,complex,chan,},%
   morekeywords=[4]{for,break,continue,range,go,goto,switch,case,fallthrough,if,%
     else,default,},%
   morekeywords=[5]{Println,Printf,Error,Print,},%
   sensitive=true,%
   morecomment=[l]{//},%
   morecomment=[s]{/*}{*/},%
   morestring=[b]',%
   morestring=[b]",%
   morestring=[s]{`}{`},%
}

\lstdefinestyle{golangStyle}{
    captionpos=b,              % sets the caption-position to bottom
    belowcaptionskip=1\baselineskip,
    breaklines=true,
    frame=none,
    numbers=none, 
    basicstyle=\footnotesize\ttfamily,
    keywordstyle=\bfseries\color{green!40!black},
    commentstyle=\itshape\color{purple!40!black},
    identifierstyle=\color{black},
    backgroundcolor=\color{white},
    language=Golang,
}


\title{TFM-FernandoMartín}
\author{Fernando Martín Canfrán}
\date{April 2024}

\begin{document}

\section{Synthetic dataset creation}

Given the confidential and private nature of bank data, it was not possible to find
any real bank datasets to perform our experiments on. In this regard, a synthetic property
graph bank dataset was built based on the Wisabi Bank Dataset\footnote{\href{https://www.kaggle.com/datasets/obinnaiheanachor/wisabi-bank-dataset}{Wisabi bank dataset on kaggle}}. It is a fictional banking dataset that 
was designed and architected by Obinna Iheanachor, and that was made publicly available in
the Kaggle platform.
 

In particular it contains 10 CSV tables. Five of them are of transaction records of five 
different states of Nigeria (Federal Capital Territory, Lagos, Kano, Enugu and Rivers State) 
that refers to transactions of cardholders in ATMs. In particular they contain 2143838 
transactions records done during the year 2022, of which 350251 are in Enugu, 159652 in 
Federal Capital Territory, 458764 in Kano, 755073 in Lagos and 420098 in Rivers.

Then, the rest of the tables are: a customers table (`customers\_lookup`) where the data
of 8819 different cardholders is gathered, an ATM table (`atm\_location lookup`) with
information of each of the 50 different locations of the ATMs, and then three remaining
tables as complement of the previous ones (`calendar lookup`, `hour lookup` and 
`transaction\_type lookup`) 
(\href{https://app.diagrams.net/#G1eAn47YR7-zPNE5KgStkA6_IJcxZRYgX8#%7B%22pageId%22%3A%22R2lEEEUBdFMjLlhIrx00%22%7D}{tables summary}).

Based on the aforementioned dataset we created our own synthetic dataset. For simplicity 
and to do it in a more stepwise manner, we are going to first create all the CSV data tables
for the nodes and for the relations in the corresponding format and then we will populate 
the Neo4j GDB with those.
The description of the creation of the CSV data tables as well as the specific details are 
described in what follows:\\

\textcolor{green}{\rule{\linewidth}{0.4mm}}

\subsubsection*{Bank}

Since a unique bank instance is considered, the values of the properties of the bank node are 
manually assigned, leaving them completely customisable. Bank node type properties consist
on the bank \emph{name}, its identifier \emph{code} and the location
of the bank headquarters, expressed in terms of \emph{latitude} and \emph{longitude}
coordinates, as seen in Table \ref{table:bank-node-properties}.
For the bank, we will generate \texttt{n} ATM and \texttt{m} Card entities. Note that 
apart from the generation of the ATM and Card node types we will also need to generate 
the relationships between the ATM and Bank entities (\texttt{belongs\_to} and \texttt{external}) 
and the Card and Bank entities (\texttt{issued\_by}).

\begin{table}[H]
    \centering
      \begin{tabular}{|l|l|}
      \hline
      \textbf{Name}        & \textbf{Description and value}                                      \\ \hline
      \texttt{name}         & Bank name                                                 \\ \hline
      \texttt{code}         & Bank identifier code                                      \\ \hline
      \texttt{loc\_latitude}  & Bank headquarters GPS-location latitude                   \\ \hline
      \texttt{loc\_longitude} & Bank headquarters GPS-location longitude                  \\ \hline
      \end{tabular}
    \caption{Bank node properties}
    \label{table:bank-node-properties}
\end{table}

\textcolor{green}{\rule{\linewidth}{0.4mm}}

\subsubsection*{ATM}

\begin{table}[H]
    \centering
    \begin{tabular}{|l|l|}
    \hline
    \textbf{Property}        & \textbf{Description}                                      \\ \hline
    \texttt{ATM\_id}      & ATM unique identifier                             \\ \hline
    \texttt{loc\_latitude}  & ATM GPS-location latitude           \\ \hline
    \texttt{loc\_longitude} & ATM GPS-location longitude          \\ \hline
    \texttt{city}         & ATM city location                         \\ \hline
    \texttt{country}      & ATM country location                       \\ \hline
    \end{tabular}
    \caption{ATM node properties}
    \label{table:atm-node-properties}
\end{table}

The bank is related with \texttt{n} ATMs. The ATM node type properties consist
on the ATM unique identifier \emph{ATM\_id}, its location, expressed in terms of 
\emph{latitude} and \emph{longitude} coordinates, as well as the \emph{city} and 
\emph{country} in which it is located, as seen in Table \ref{table:atm-node-properties}.
\textcolor{red}{Note that the last two properties are somehow redundant, considering
that location coordinates are already included. In any case both properties are left
since their inclusion provide a more human-understandable way to easily realise about
its location}.

\begin{itemize}
  \item Poner que hay dos tipos de ATM en base a su relacion con el banco -internal
  y external
  \item Explicar lo de como se generan en base a la distribucion de las locations 
  en wisabi...
\end{itemize}

The generation of \texttt{n} ATMs for the bank is done following
the geographical distribution of the locations of the ATMs in the wisabi dataset. On it there are 50 ATMs
locations distributed along Nigerian cities. The distribution of the ATMs matches the
importance of the location since the number of ATM locations is larger in the most populated 
Nigerian cities (30\% of the ATM locations are in the city of Lagos, then the 20\% in 
Kano...) Therefore, for generating a new ATM location first we select uniformly at 
random an ATM location/city from the wisabi dataset, which is directly assigned as 
\texttt{city} and \texttt{country} to the ATM instance, and generate new random 
geolocation coordinates inside a constructed bounding box of this city location to set 
as the \texttt{loc\_latitude} and \texttt{loc\_longitude} of the ATM.

\begin{tcolorbox}
\textcolor{red}{$\Rightarrow$ Okay like this: }
\textcolor{blue}{Note that we do not take into account for the density distribution of the 
ATMs of the wisabi dataset that, for each ATM location of the dataset, we have \texttt{x} 
number of atms.}

Aspects to explain:
\begin{itemize}
  \item Geographical distribution of the ATMs based on the geographical distribution 
  of the locations of the Wisabi ATMs.
  \item Do a plot of the geographical dist of the positions of the ATMs of the wisabi dataset.
\end{itemize}
\end{tcolorbox}

\subsubsection*{Card}

\begin{itemize}
  \item Explicar las properties con la tabla y de la forma que se hizo descriptiva
  para ATM y Bank.
\end{itemize}

\begin{itemize}
\item[-] number\_id: Unique identifier of the card.
\item[-] client\_id: Unique identifier of the client.
\item[-] expiration: Validity expiration date of the card.
\item[-] CVC: Card Verification Code.
\item[-] extract\_limit: Limit amount of money extraction associated with the card.
\item[-] loc\_latitude: Client address GPS-location latitude.
\item[-] loc\_longitude: Client address GPS-location longitude.
\end{itemize}

\begin{tcolorbox}
  Aspects to explain:
  \begin{itemize}
    \item Extended behavior fields: to not only withdrawal.
    \item Location: Explain the 2 options we have developed and the one used so far.
    \item Explain how the behavior of the card is generated based on a random
    selected wisabi customer.
    \item Extract\_limit: explain how and why?
  \end{itemize}
\end{tcolorbox}

For the Card entity generation there are some different aspects that are worth to mention:
\begin{itemize}
    \begin{tcolorbox}
      \item First, in relation with the card and client identifiers (\texttt{number\_id}, 
      \texttt{client\_id}), \textcolor{blue}{for the moment we define that each client has 
      1 card, later this can be modified.\\}
      \textcolor{red}{$\Rightarrow$ For the moment 1 client : 1 card. So far, 
      for the kind of frauds we are considering this is the easier approach. In a 
      future, if needed, it could be generated the case that 1 client has n cards} 
    \end{tcolorbox}

    \item \texttt{Expiration} and \texttt{CVC} fields: they are not relevant, could be empty 
    fields indeed or for all the Cards the same values. For simplicity and completeness we 
    chose them to be the same values for all the Cards.
    \begin{tcolorbox}
      Expiration: set for completeness the same date in all of them but in a far future!
    \end{tcolorbox}
    \item \textcolor{blue}{Behavior: For each Card object to be generated, we assign it a 
    \textit{behavior} based on the transaction behavior of a randomly selected wisabi 
    customer. The behavior is gathered from the transaction history of the customer on 
    the wisabi dataset. This will be particularly useful later for the generation of the 
    synthetic transactions, so that for each of the cards their transactions can be 
    simulated based on this gathered behavior.}
    In particular the customer \textit{behavior} refers to: 
    \begin{itemize}
        \item \texttt{extract\_limit}: maximum normal amount that a card can extract. 
        $\textcolor{blue}{\texttt{amount\_avg} * 5}$.
        \item \texttt{amount\_avg}: extracted amount average of the customer on a transaction.
        \item \texttt{amount\_std}: extracted amount standard deviation of the customer 
        on a transaction.
        \item \texttt{withdrawal\_day}: average number of transactions per day of the 
        customer. \textcolor{red}{Withdrawals only for the moment!}.
    \end{itemize}
    \begin{tcolorbox}
      \item New added behavior: interesting for the generation of transactions that are
      not only withdrawals: balance inquiries, deposits, transfers.
    \end{tcolorbox}
    Note that all this fields are additionally added to the Card CSV records.
    Some additional remarks:
    \begin{itemize}
        \item \textcolor{red}{For the moment we only consider the \textit{withdrawal} type of transaction in the behavior. However \textit{transfer} and \textit{deposit} could be also considered.}
        % retirada de dinero 
        \item \textcolor{red}{This behavior is gathered from one random customer of the wisabi dataset per each of the Cards, so that we have more variability. However, we could also assign the same behavior to all the clients, and this behavior be like a summary of all the wisabi dataset clients behavior. Also the behavior could be assigned drawning it from taylored distributions selected by us, in a more customizable manner.}   
    \end{itemize}
    \item Location (\texttt{loc\_latitude}, \texttt{loc\_longitude}): Two possible options in this case:
    \begin{itemize}
        \item 1. Assign a random location of the usual ATM city/location of the random selected wisabi customer. This way, we maintain the geographical distribution of the wisabi customers.
        \item 2. Assign a random location of the city/location of a random ATM of the newly generated ATMs objects. (* For the moment)
    \end{itemize}
\end{itemize}

\section{Indexing}

Useful for ensuring efficient lookups and obtaining a better performance as the database 
scales.

$\rightarrow$ indexes will be created on those properties of the entities on which the 
lookups are going to be mostly performed; specifically in our case:
\begin{itemize}
  \item Bank: \texttt{code} ?
  \item ATM: \texttt{ATM\_id}
  \item Card: \texttt{number\_id}
\end{itemize}

Why on these ones?

$\rightarrow$ Basically the volatile relations / transactions only contain this information,
which is the minimal information to define the transaction. This is the only information that
the engine recieves from a transaction, and it is the one used to retrieve additional information - the complete information details of the ATM and Card nodes on the complete
stable bank database. Therefore these parameters/fields (look for the specific correct
word on the PG world) are the ones used to retrieve / query the PG. 

By indexing or applying a unique constraint on the node properties, queries related to these entities can be optimized, ensuring efficient lookups and better performance as the database scales.

From Neo4j documentation:
\begin{tcolorbox}
  An index is a copy of specified primary data in a Neo4j database, such as nodes, relationships, or properties. The data stored in the index provides an access path to the data in the primary storage and allows users to evaluate query filters more efficiently (and, in some cases, semantically interpret query filters). In short, much like indexes in a book, their function in a Neo4j graph database is to make data retrieval more efficient.
\end{tcolorbox}

Some references on indexing:
\begin{itemize}
  \item \href{https://neo4j.com/docs/cypher-manual/current/indexes/search-performance-indexes/overview/}{Search-performance indexes}
  \item \href{https://neo4j.com/docs/cypher-manual/current/indexes/search-performance-indexes/using-indexes/}{The impact of indexes on query performance}
  \item \href{https://neo4j.com/docs/cypher-manual/current/indexes/search-performance-indexes/managing-indexes/}{Create, show, and delete indexes}
\end{itemize}

Okay... but before diving deeper...:

\textbf{To Index or Not to Index?}
\begin{tcolorbox}
When Neo4j creates an index, it creates a redundant copy of the data in the database. Therefore using an index will result in more disk space being utilized, plus slower writes to the disk.

Therefore, you need to weigh up these factors when deciding which data/properties to index.

Generally, it's a good idea to create an index when you know there's going to be a lot of data on certain nodes. Also, if you find queries are taking too long to return, adding an index may help.
\end{tcolorbox}

From \href{https://www.quackit.com/neo4j/tutorial/neo4j_create_an_index_using_cypher.cfm#google_vignette}{another tutorial on indexing in neo4j}

\end{document}