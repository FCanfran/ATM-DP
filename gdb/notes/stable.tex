\documentclass{article}
\usepackage[utf8]{inputenc}
\usepackage{graphicx} % Required for inserting images
\usepackage{hyperref}
\usepackage{subcaption}
\usepackage{float}
\usepackage{tcolorbox}
\usepackage{amsmath}
\usepackage{amssymb}
\usepackage{listings}% http://ctan.org/pkg/listings}
\usepackage{algorithm}
\usepackage{algorithmic}
\usepackage[toc,page]{appendix}
\usepackage[backend=biber]{biblatex}
\usepackage{multicol}
\usepackage{siunitx}
\usepackage{comment}
\usepackage{xcolor}
\usepackage{caption}
\usepackage{tikz}
\usetikzlibrary{shapes.geometric, arrows}

\tikzstyle{startstop} = [rectangle, rounded corners, 
minimum width=3cm, 
minimum height=1cm,
text centered, 
draw=black, 
fill=red!30]

\tikzstyle{io} = [trapezium, 
trapezium stretches=true, % A later addition
trapezium left angle=70, 
trapezium right angle=110, 
minimum width=3cm, 
minimum height=1cm, text centered, 
draw=black, fill=blue!30]

\tikzstyle{process} = [rectangle, 
minimum width=3cm, 
minimum height=1cm, 
text centered, 
text width=3cm, 
draw=black, 
fill=orange!30]

\tikzstyle{decision} = [diamond, 
minimum width=3cm, 
minimum height=1cm, 
text centered, 
draw=black, 
fill=green!30]
\tikzstyle{arrow} = [thick,->,>=stealth]

% To delete lstlisting caption "Listing x"
%\captionsetup[lstlisting]{labelformat=empty}

\lstdefinestyle{myStyle}{
    belowcaptionskip=1\baselineskip,
    breaklines=true,
    frame=none,
    numbers=none, 
    basicstyle=\footnotesize\ttfamily,
    keywordstyle=\bfseries\color{green!40!black},
    commentstyle=\itshape\color{purple!40!black},
    identifierstyle=\color{black},
    backgroundcolor=\color{white},
}

\lstdefinestyle{cypherStyle}{
    backgroundcolor=\color{white},   % choose the background color
    basicstyle=\footnotesize\ttfamily,        % the size of the fonts that are used for the code
    commentstyle=\itshape\color{purple!40!black},
    keywordstyle=\bfseries\color{green!40!black},
    breakatwhitespace=false,         % sets if automatic breaks should only happen at whitespace
    breaklines=true,                 % sets automatic line breaking
    captionpos=b,                    % sets the caption-position to bottom
    commentstyle=\color{gray},    % comment style
    deletekeywords={},            % if you want to delete keywords from the given language
    escapeinside={\%*}{*)},          % if you want to add LaTeX within your code
    extendedchars=true,              % lets you use non-ASCII characters; for 8-bits encodings only, does not work with UTF-8
    %firstnumber=1000,                % start line enumeration with line 1000
    frame=none,                    % adds a frame around the code
    keepspaces=true,                 % keeps spaces in text, useful for keeping indentation of code (possibly needs columns=flexible)
    language=SQL,                    % the language of the code
    morekeywords={*,IF, REQUIRE, FOR, IS, LOAD, CSV, WITH, HEADERS, MERGE, toFloat, toInteger, date},            % if you want to add more keywords to the set
    numbers=none,                    % where to put the line-numbers; possible values are (none, left, right)
    numbersep=5pt,                   % how far the line-numbers are from the code
    numberstyle=\tiny\color{mygray}, % the style that is used for the line-numbers
    rulecolor=\color{black},         % if not set, the frame-color may be changed on line-breaks within not-black text (e.g. comments (green here))
    showspaces=false,                % show spaces everywhere adding particular underscores; it overrides 'showstringspaces'
    showstringspaces=false,          % underline spaces within strings only
    showtabs=false,                  % show tabs within strings adding particular underscores
    stepnumber=1,                    % the step between two line-numbers. If it's 1, each line will be numbered
    stringstyle=\ttfamily,     % string literal style
    tabsize=2,                       % sets default tabsize to 2 spaces
}

%% Golang definition for listings
%% http://github.io/julienc91/lstlistings-golang
%%
\lstdefinelanguage{Golang}%
  {morekeywords=[1]{package,import,func,type,struct,return,defer,panic,%
     recover,select,var,const,iota,},%
   morekeywords=[2]{string,uint,uint8,uint16,uint32,uint64,int,int8,int16,%
     int32,int64,bool,float32,float64,complex64,complex128,byte,rune,uintptr,%
     error,interface},%
   morekeywords=[3]{map,slice,make,new,nil,len,cap,copy,close,true,false,%
     delete,append,real,imag,complex,chan,},%
   morekeywords=[4]{for,break,continue,range,go,goto,switch,case,fallthrough,if,%
     else,default,},%
   morekeywords=[5]{Println,Printf,Error,Print,},%
   sensitive=true,%
   morecomment=[l]{//},%
   morecomment=[s]{/*}{*/},%
   morestring=[b]',%
   morestring=[b]",%
   morestring=[s]{`}{`},%
}

\lstdefinestyle{golangStyle}{
    captionpos=b,              % sets the caption-position to bottom
    belowcaptionskip=1\baselineskip,
    breaklines=true,
    frame=none,
    numbers=none, 
    basicstyle=\footnotesize\ttfamily,
    keywordstyle=\bfseries\color{green!40!black},
    commentstyle=\itshape\color{purple!40!black},
    identifierstyle=\color{black},
    backgroundcolor=\color{white},
    language=Golang,
}


\title{TFM-FernandoMartín}
\author{Fernando Martín Canfrán}
\date{April 2024}

\begin{document}

\section{Synthetic dataset creation}

Given the confidential and private nature of bank data, it was not possible to find
any real bank datasets. In this regard, a synthetic property
graph bank dataset was built based on the \emph{Wisabi Bank Dataset}\footnote{\href{https://www.kaggle.com/datasets/obinnaiheanachor/wisabi-bank-dataset}{Wisabi bank dataset on kaggle}}. It is a fictional banking dataset that was made publicly available in
the Kaggle platform.

This synthetic bank dataset was considered of interest as a base for the synthetic bank
database that we wanted to develop. The interest to use this bank dataset as a base was mainly because of its size: it contains 8819 different customers, 50 different ATM locations and 2143838 transactions records of the different customers during a full year (2022). Additionally, it provides good heterogenity on the different kind of transactions: withdrawals, deposits, balance inquiries and transfers.

The main uses of this bank dataset are the obtention of a geographical distribution for
the locations of our generated ATMs and the construction of a card/client \emph{behavior},
for which the data of the \emph{Wisabi Bank Dataset} will be used.

\paragraph{Details of the \emph{Wisabi Bank Dataset}}
The \emph{Wisabi Bank Dataset} consists on ten CSV tables. Five of them are of transaction records of five different states of Nigeria (Federal Capital Territory, Lagos, Kano, Enugu and Rivers State) that refers to transactions of cardholders in ATMs. In particular they contain 2143838 transactions records done during the year 2022, of which 350251 are in Enugu, 159652 in Federal Capital Territory, 458764 in Kano, 755073 in Lagos and 420098 in Rivers. Then, the rest of the tables are: a customers table (`customers\_lookup`) where the data
of 8819 different cardholders is gathered, an ATM table (`atm\_location lookup`) with
information of each of the 50 different locations of the ATMs, and then three remaining
tables as complement of the previous ones (`calendar lookup`, `hour lookup` and 
`transaction\_type lookup`) 
(\href{https://app.diagrams.net/#G1eAn47YR7-zPNE5KgStkA6_IJcxZRYgX8#%7B%22pageId%22%3A%22R2lEEEUBdFMjLlhIrx00%22%7D}{tables summary}).

In what follows we give the details on the generation of the instances of our static database
entities.
For simplicity and to do it in a more stepwise manner, we are going to first create all the CSV data tablesfor the nodes and for the relations in the corresponding format and then we will populate the Neo4j GDB with them.

\subsubsection*{Bank}

Since a unique bank instance is considered, the values of the properties of the bank node are 
manually assigned, leaving them completely customisable. Bank node type properties consist
on the bank \emph{name}, its identifier \emph{code} and the location
of the bank headquarters, expressed in terms of \emph{latitude} and \emph{longitude}
coordinates, as seen in Table \ref{table:bank-node-properties}.
For the bank, we will generate \texttt{n} ATM and \texttt{m} Card entities. Note that 
apart from the generation of the ATM and Card node types we will also need to generate 
the relationships between the ATM and Bank entities (\texttt{belongs\_to} and \texttt{external}) 
and the Card and Bank entities (\texttt{issued\_by}).

\begin{table}[H]
    \centering
      \begin{tabular}{|l|l|}
      \hline
      \textbf{Name}        & \textbf{Description and value}                                      \\ \hline
      \texttt{name}         & Bank name                                                 \\ \hline
      \texttt{code}         & Bank identifier code                                      \\ \hline
      \texttt{loc\_latitude}  & Bank headquarters GPS-location latitude                   \\ \hline
      \texttt{loc\_longitude} & Bank headquarters GPS-location longitude                  \\ \hline
      \end{tabular}
    \caption{Bank node properties}
    \label{table:bank-node-properties}
\end{table}

\subsubsection*{ATM}

\begin{table}[H]
    \centering
    \begin{tabular}{|l|l|}
    \hline
    \textbf{Property}        & \textbf{Description}                                      \\ \hline
    \texttt{ATM\_id}      & ATM unique identifier                             \\ \hline
    \texttt{loc\_latitude}  & ATM GPS-location latitude           \\ \hline
    \texttt{loc\_longitude} & ATM GPS-location longitude          \\ \hline
    \texttt{city}         & ATM city location                         \\ \hline
    \texttt{country}      & ATM country location                       \\ \hline
    \end{tabular}
    \caption{ATM node properties}
    \label{table:atm-node-properties}
\end{table}

The bank operates \texttt{n} ATMs, categorized in:

\begin{itemize}
  \item Internal ATMs: ATMs owned and operated by the bank. They are fully integrated within the
  bank's network. 
  \item External ATMs: These ATMs, while not owned by the bank, are still accessible for the bank
  customers to perform transactions.
\end{itemize}

Both types of ATMs are considered to be of the same type of ATM node. Their difference
is modeled as their relation with the bank instance: \texttt{belongs\_to} for the internal ATMs and \texttt{external} for the external ATMs, having:
$$\texttt{n = n\_internal + n\_external}$$

where \texttt{n\_internal} is the number of internal ATMs owned by the bank and \texttt{n\_external}
is the number of external ATMs that are accesible to the bank.


The ATM node type properties consist on the ATM unique identifier \emph{ATM\_id}, its location, expressed in terms of 
\emph{latitude} and \emph{longitude} coordinates, and the \emph{city} and 
\emph{country} in which it is located, as seen in Table \ref{table:atm-node-properties}.
\textcolor{red}{Note that the last two properties are somehow redundant, considering
that location coordinates are already included. In any case both properties are left
since their inclusion provide a more human-understandable way to easily realise about
the location of the ATMs}.

The generation of \texttt{n} ATMs for the bank is done following
the geographical distribution of the locations of the ATMs in the \emph{Wisabi Bank Dataset}. 
On this dataset there are 50 ATMs locations distributed along Nigerian cities. 
Note that for each of these ATMs locations, there can be more than one ATM.
However, this is not taken into account and only one ATM per location is assumed for the 
distribution.\\
\textcolor{red}{$\Rightarrow$ Put a plot of the distribution of the ATM locations}\\
This distribution of the ATMs matches the relevance of the location in terms of its 
population, since the number of ATM locations is larger in the most populated 
Nigerian cities (30\% of the ATM locations are in the city of Lagos, then the 20\% in 
Kano...).
Therefore, for the generation of the location of each of the \texttt{n} ATMs, the location/city of an ATM selected uniformly at random from the \emph{Wisabi Bank Dataset} is assigned as \emph{city} and \emph{country}. Then, new random geolocation coordinates 
inside a bounding box of this city location are set as the \emph{loc\_latitude} and \emph{loc\_longitude} exact coordinates of the ATM. \\
Finally, as the ATM unique identifier \emph{ATM\_id} it is assigned a different code depending on the ATM internal or external category: 

\[
\emph{ATM\_id} =
\begin{cases} 
bank\_code + "-" + i & 0 \leq i < \texttt{n\_internal } \text{if internal ATM}  \\
EXT + "-" + i & 0 \leq i < \texttt{n\_external } \text{if external ATM}
\end{cases}
\]

\subsubsection*{Card}

\begin{table}[H]
    \centering
    \begin{tabular}{|l|l|}
    \hline
    \textbf{Name}        & \textbf{Description and value}                                          \\ \hline
    \texttt{number\_id}   & Card unique identifier                               \\ \hline
    \texttt{client\_id}   & Client unique identifier                               \\ \hline
    \texttt{expiration}   & Card validity expiration date                      \\ \hline
    \texttt{CVC}          & Card Verification Code                                      \\ \hline
    \texttt{extract\_limit} & Card money amount extraction limit    \\ \hline
    \texttt{loc\_latitude}  & Client's habitual address GPS-location latitude                         \\ \hline
    \texttt{loc\_longitude} & Client's habitual address GPS-location longitude                        \\ \hline
    \end{tabular}
    \caption{Card node properties}
    \label{table:card-node-properties}
\end{table}

\begin{itemize}
  \item Explicar las properties con la tabla y de la forma que se hizo descriptiva
  para ATM y Bank.
\end{itemize}

The bank manages a total of \texttt{m} cards. The Card node type properties, as depicted in Table
\ref{table:card-node-properties}, consist on the card unique 
identifier \emph{number\_id}, the associated client unique identifier \emph{client\_id}, as well
as the coordinates of the associated client habitual residence address \emph{loc\_latitude} and 
\emph{loc\_longitude}. Additionally it contains the card validity expiration date \emph{expiration}
and the Card Verification Code, \emph{CVC}.\\

\textcolor{red}{$\Rightarrow ?$ Finally, it contains the property \emph{extract\_limit}
which represents the limit on the amount of money it can be extracted with the card on a single 
extraction/day?}

\textcolor{red}{$\Rightarrow ?$ Include in the card properties the properties related with the
gathered behavior for the card: \emph{withdrawal\_day}, \emph{transfer\_day}, 
\emph{withdrawal\_avg}...}

\begin{tcolorbox}
  Aspects to explain:
  \begin{itemize}
    \item Extract\_limit: explain how and why?
  \end{itemize}
\end{tcolorbox}

\begin{itemize}
\item Card and client identifiers:
so far, although for completeness the \emph{client\_id} is included in the properties of the Card node type, note that for simplicity it could be ignored, since due to the purposes of our work, a \emph{one-to-one} relationship between card and client 
is assumed, meaning that each card is uniquely associated with a single client, and that a client
can possess only one card. Therefore, the \emph{client\_id} is not relevant so far, but is included
in case the database model is extended to allow clients have multiple cards or cards belonging to 
multiple different clients. For each generated Card instance these identifiers are set as:

\[
\begin{cases} 
number\_id = \text{c-}bank\_code\text{-}i \\
client\_id = i 
\end{cases}
0 \leq i < \texttt{m}
\]

\item \texttt{Expiration} and \texttt{CVC} properties: they are not relevant, could be empty 
  value properties indeed or a same toy value for all the cards. For completeness the  
  same values are given for all the cards: $\texttt{Expiration} = \text{2050-01-17}$, $\texttt{CVC} = 999$.

\item Client's habitual address location (\texttt{loc\_latitude}, \texttt{loc\_longitude}): two possible options were designed to define the client habitual residence address. In both 
cases they are random 
coordinates drawn from a bounding box of a location/city. The difference is on to do the selection of the location/city:

  \begin{enumerate}
      \item Wisabi customers selection: Take the city/location of the habitual ATM of a random selected \emph{Wisabi} database customer. Note that in the \emph{Wisabi Bank Dataset} customers contain an identifier
      of their usual ATM, more in particular, the dataset is designed in such a way that customers
      only perform operations in the same ATM.
      With this approach, we maintain the geographical distribution of the \emph{Wisabi} customers.
      \item Generated ATMs selection: Take the city/location of a random ATM of the \texttt{n} generated ATMs. This method is the one utilized so far.
  \end{enumerate}

\item[$\circ$]\textbf{\emph{Behavior}}: It contains relevant attributes that will be of special interest when performing the 
generation of the synthetic transactions of each of the cards. The defined \emph{behavior}
parameters are shown in Table \ref{table:behavior-parameters}. 

\begin{table}[H]
    \centering
    \begin{tabular}{|l|l|}
        \hline
        \textbf{Behavior Property} & \textbf{Description} \\ 
        \hline
        $\mathsf{amount\_avg\_withdrawal}$ & Withdrawal amount mean\\ 
        \hline
        $\mathsf{amount\_std\_withdrawal}$ & Withdrawal amount standard deviation \\ 
        \hline
        $\mathsf{amount\_avg\_deposit}$ & Deposit amount mean \\ 
        \hline
        $\mathsf{amount\_std\_deposit}$ & Deposit amount standard deviation\\ 
        \hline
        $\mathsf{amount\_avg\_transfer}$ & Transfer amount mean \\ 
        \hline
        $\mathsf{amount\_std\_transfer}$ & Transfer amount standard deviation \\ 
        \hline
        $\mathsf{withdrawal\_day}$ & Average number of withdrawal operations per day \\ 
        \hline
        $\mathsf{deposit\_day}$ & Average number of deposit operations per day \\ 
        \hline
        $\mathsf{transfer\_day}$ & Average number of transfer operations per day \\ 
        \hline
        $\mathsf{inquiry\_day}$ & Average number of inquiry operations per day \\ 
        \hline
    \end{tabular}
    \caption{\emph{Behavior} properties}
    \label{table:behavior-properties}
\end{table}


For each card, its \emph{behavior} parameters are gathered from the operations history of a randomly selected customer on the \emph{Wisabi Bank Dataset}, from which we can access the operations log of $8819$ different customers for one year time interval. On it, there are four different types of operations that a customer can perform: withdrawal, deposit, balance inquiry and transaction. The parameters
for the \emph{behavior} gather information about these four different types of operations.

\textcolor{gray}{Note that all these \emph{behavior} parameters are added as additional fields of the CSV generated card instances, so, as mentioned, they can later be utilized for the generation of the synthetic
transactions.}

Another possible way to assign the \emph{behavior} parameters could be the assignation
of the same behavior to all of the card instances. However, this method will provide less variability in
the generation of the synthetic transactions than the aforementioned method. 
Nevertheless, other taylored generation methods to generate different \emph{behavior} for 
each the cards could also be considered to similarly obtain this
variability.

\item \textcolor{red}{\texttt{extract\_limit}: $\texttt{amount\_avg\_withdrawal} * 5$}
\end{itemize}

\textcolor{green}{\rule{\linewidth}{0.4mm}}

\section{Indexing}

Useful for ensuring efficient lookups and obtaining a better performance as the database 
scales.

$\rightarrow$ indexes will be created on those properties of the entities on which the 
lookups are going to be mostly performed; specifically in our case:
\begin{itemize}
  \item Bank: \texttt{code} ?
  \item ATM: \texttt{ATM\_id}
  \item Card: \texttt{number\_id}
\end{itemize}

Why on these ones?

$\rightarrow$ Basically the volatile relations / transactions only contain this information,
which is the minimal information to define the transaction. This is the only information that
the engine recieves from a transaction, and it is the one used to retrieve additional information - the complete information details of the ATM and Card nodes on the complete
stable bank database. Therefore these parameters/fields (look for the specific correct
word on the PG world) are the ones used to retrieve / query the PG. 

By indexing or applying a unique constraint on the node properties, queries related to these entities can be optimized, ensuring efficient lookups and better performance as the database scales.

From Neo4j documentation:
\begin{tcolorbox}
  An index is a copy of specified primary data in a Neo4j database, such as nodes, relationships, or properties. The data stored in the index provides an access path to the data in the primary storage and allows users to evaluate query filters more efficiently (and, in some cases, semantically interpret query filters). In short, much like indexes in a book, their function in a Neo4j graph database is to make data retrieval more efficient.
\end{tcolorbox}

Some references on indexing:
\begin{itemize}
  \item \href{https://neo4j.com/docs/cypher-manual/current/indexes/search-performance-indexes/overview/}{Search-performance indexes}
  \item \href{https://neo4j.com/docs/cypher-manual/current/indexes/search-performance-indexes/using-indexes/}{The impact of indexes on query performance}
  \item \href{https://neo4j.com/docs/cypher-manual/current/indexes/search-performance-indexes/managing-indexes/}{Create, show, and delete indexes}
\end{itemize}

Okay... but before diving deeper...:

\textbf{To Index or Not to Index?}
\begin{tcolorbox}
When Neo4j creates an index, it creates a redundant copy of the data in the database. Therefore using an index will result in more disk space being utilized, plus slower writes to the disk.

Therefore, you need to weigh up these factors when deciding which data/properties to index.

Generally, it's a good idea to create an index when you know there's going to be a lot of data on certain nodes. Also, if you find queries are taking too long to return, adding an index may help.
\end{tcolorbox}

From \href{https://www.quackit.com/neo4j/tutorial/neo4j_create_an_index_using_cypher.cfm#google_vignette}{another tutorial on indexing in neo4j}

\end{document}