\newpage
\section{Proposal}
\label{sec:proposal}

Defining and implementing a continuous query engine requires to address many different problems, each of different nature. 
In addition, the proof of concept we intend to provide has itself its own complications.
We address all of them in this section.
%

\subsection{Modeling and Implementing the Continuous Query Engine}
To define a proper architecture for a continuous query engine is one of the most challenging activities of our work. Among other tasks, this comprises: to define a graph-based query language expressive enough that allows capturing the different patterns representing anomalous queries; to establish the algorithms for identifying the patterns associated to anomalous queries; to choose and manage the right windowing approach and other features related to distributed query-evaluation; to deal with the evaluation of many continuous queries simultaneously; to evaluate the suitability of the implementation language, tools and the proper system configuration. Figure \ref{fig:DP_ATM} depicts a preliminary architecture for a continuous query engine for detecting anomalous ATM transactions, $\mathsf{DP_{ATM}}$.
Figure \ref{fig:architecture-usage} depicts an abstraction of the architecture of the Continuous Query Engine that we propose and show how this system can be used to prevent ATM transactions frauds combining it with a double check mechanism. 
%
\begin{figure}[H]
         \centering
         \hspace*{-0.8cm}
         \includegraphics[scale = 0.32]{images/architectureCQE.png}
         \caption{Abstraction of the Continuous Query Engine for detecting anomalous 
         ATM transactions based on the dynamic pipeline computational model. 
         In this abstract architecture of the Continuous Query  all its components as well as its expected usage are shown. }
         \label{fig:architecture-usage}
\end{figure}
%
%Still we are working on the design and implementation of this model. 
As we said, we propose a solution that follows the Dynamic Computational Approach \cite{DP-pasarella2024computational}. Briefly speaking, in this approach, \emph{stages} are processes that execute tasks concurrently/in-parallel. The multiset underlying the input data stream is partitioned \cite{bender1974partitions} and distributed along filters according to a grouping relationship, usually based on filters' parameters. Each filter applies the same function to its block of data (stored as its state). Accordingly, the  $\mathsf{DP_{ATM}}$ algorithm is specified as follows: During a pre-defined time interval window, when an \textsf{interaction} $\mathsf{e}$ (together its properties' values)  arrives to the $\mathsf{DP_{ATM}}$, the stage $\mathsf{S_r}$ register it into a standard transactional log file. Then, $\mathsf{S_r}$ passes $\mathsf{e}$ to the next stage. If there exists a filter parameterized with the value of the property \textsf{number} of the Card vertex that is incident to  $\mathsf{e}$, this filter keeps $\mathsf{e}$ in its state. In this way, filters' states store  subgraphs induced by the edges in the volatile subgraph. Notice that these sets of edges in each filter correspond to blocks of the (multiset) input data stream.  Otherwise, the filter passes $\mathsf{e}$ to the next stage. The task/function of each filter is to decide if there is a match with (some of) the continuous query pattern(s) evaluated by the engine $\mathsf{DP_{ATM}}$ by means of the graph that it stores and the information retrieved from the stable PG to identify patterns and solve constraints. This is, indeed, the way to evaluate continuous queries. In case of matching a pattern, filters emit an alert reporting the finding. Hence, answers are the detected anomalies and they are emitted as they are obtained in filters. When answers arrive to $\mathsf{S_k}$, this stage  post-processes  and output  them. In addition, $\mathsf{S_k}$ maintains an answer log file. The fact that an \textsf{interaction} arrives to $\mathsf{G}$ means that there were not previous interactions having the same value of Card property \textsf{number} and thus, a new filter parameterized with this new value is spawned. When the time interval window is over, the $\mathsf{DP_{ATM}}$ is, in some sense, reset according to the given window policy. Note that the window policy must take into account stored data that might be valid in between two windows and handle the transition properly.
%
\subsection{Defining Anomalous Patterns of Transactions}\label{sub:anomalouspatterns}
It is not trivial to establish what is and in which circumstances a transaction can be considered anomalous. Based on a work that have addressed this characterization \cite{magdalena2021artificial} we intend to find a proper characterization and then define the graph patterns associated to these anomalies. The exact topology of an anomaly will depend on its own nature. Figure \ref{fig:constinuousPGb} depicts an example characterizing a possible card cloning, among many other possibilities. For instance, using a (stolen) card many times over a period of time at different ATMs to withdraw small amounts. In this latter case, there will arrive to the evolving PG many volatile (interaction) edges having the same card vertex and different ATM vertices. There could also be patterns related with frequent/very high expenses; transactions  located in an ATM out of the threshold distance of the usual/registered address of the card holder and so on.
Moreover, definition of patterns can be beyond ATM transactions by considering Point-Of-Sale (POS) or online card transactions.
%
\iffalse
\subsubsection*{Fraud Patterns Definition}

It is not trivial to establish what is and in which circumstances an ATM transaction can be considered anomalous. Based on a work that have addressed this characterization \cite{FP-magdalena2021artificial} we intend to find a proper characterization and then define the graph patterns associated to these anomalies. The exact topology of an anomaly will depend on its own nature. Moreover, definition of patterns can be beyond ATM transactions by considering online card transactions. In what follows, we propose a characterization of some possible anomalous patterns of ATM transactions and the definition of their associated PG graph patterns. 
\fi
%
\begin{enumerate}
\renewcommand{\labelenumi}{\Roman{enumi}.} % Roman numerals for the list
    \item Card cloning characterization
    \item Lost-and-stolen card characterization
    %\item Anomalous amount of withdrawals in a time period
    \item Other possible fraud scenarios
\end{enumerate}


\paragraph{I - Card Cloning Characterization\\\\}

\emph{Card cloning} can be defined as "a kind of fraud in which information on a card used for a transaction is covertly and illegally duplicated. 
Basically, it’s a process thieves use to copy the information on a transaction card without stealing the physical card itself. 
This information is then copied onto a new or reformatted card, allowing criminals to use it to make fraudulent purchases or gain unauthorized access to a person’s accounts" \cite{FP-unit21_card_cloning}.\\

There are many possible ways to detect a card cloning scenario, among others, the analysis of the customer's transaction data to construct typical transaction behaviors so to be able to detect uncommon transaction behaviors. However, in our work we propose an alternative possible method based on a PG graph pattern detection.\\

The method consists on detecting abnormal card-ATM activity of the same card at different ATMs taking place within an unfeasible time distance difference. That is, when a transaction is made at an ATM, and after that, another transaction is initiated with the same card at a different ATM, such that the distance between the two is impossible to be covered within the time between the transactions.
The detection of this anomalous scenario is represented on the PG graph pattern of Figure \ref{img:graphPattern-1}. 

\begin{figure}[H]
  \centering
  \includegraphics[scale = 0.8]{images/2-QueryModel/graphPattern-1.png}
  \caption{Property graph pattern - Card cloning characterization}
  \label{img:graphPattern-1}
\end{figure}

The pattern consists on a \texttt{Card} entity $x_1$, having two \texttt{interaction} relations $y_1$ and $y_2$ with two different \texttt{ATMs} $x_2$ and $x_3$, respectively, such that the time difference between the ending time of the first \texttt{interaction} $y_1.\textit{end}$ and the starting time of the second \texttt{interaction} $y_2.\textit{start}$, is not sufficient to cover the minimum time needed to travel from the first to the second \texttt{ATM} location $T_{min}(x_2.\textit{location}, x_3.\textit{location})$. As a whole:

$$
\small
x2.id \ne x3.id \ \land \ y_2.\textit{start} - y_1.\textit{end} < T_{min}(x_2.\textit{location}, x_3.\textit{location})
$$

where $x_2.\textit{location}$ represents the location coordinates pair of the $x_2$ ATM: $x_2.location = (x_2.loc\_latitude, x_2.loc\_longitude)$. Same for the \texttt{ATM} $x_3$.\\

An example of this kind of anomalous card-ATM interaction, could be one as represented on Figure \ref{img:graphPattern-1-Example}, in which an ATM interaction with a certain card is finished at time 22:14 in Barcelona, and then another interaction with that same card starts at time 22:56 of that same day in Madrid. Clearly this should be reported as this kind of anomalous scenario since it is impossible, for the time being, to cover the distance between these two cities in that time interval.

\begin{figure}[H]
  \centering
  \includegraphics[scale = 0.7]{images/2-QueryModel/FP1-Example.png}
  \caption{Card cloning characterization - an example}
  \label{img:graphPattern-1-Example}
\end{figure}

\paragraph{II - Lost-and-Stolen Card Characterization\\\\}

"Lost-and-stolen card is the fraud scenario produced when a card is physically stolen or is lost, and is then used by a criminal, posing as you, to obtain goods and services" \cite{FP-lost-and-stolen-americanexpress2025}.\\

A possible way that we propose to detect this kind of fraud scenario is through the tracking of a typical behavior that it is produced when the card is used by the criminal. That is, when obtained, the fraudster tries to do as many as possible money withdrawals in different ATMs before the owner of the card become aware of the loss of the card and asks the bank to freeze it. The detection of this kind of fraud scenario is modeled with a PG graph pattern as the one represented in Figure \ref{img:graphPattern-2}.

\begin{figure}[H]
  \centering
  \includegraphics[scale = 0.7]{images/2-QueryModel/graphPattern-2.png}
  \caption{Property graph pattern - Lost-and-stolen card characterization. The interactions $y_1,...,y_j$ depicted are of the \emph{type} withdrawal}
  \label{img:graphPattern-2}
\end{figure}

On it we define a \texttt{Card} entity $x_1$ having a number $k$ of \texttt{interactions} $y$ at different \texttt{ATMs} $x_2 ... x_k$ within a time interval $[t_i, t_i + \Delta]$, where $t_i = y_1.\textit{start}$ and $t_i + \Delta = y_j.\textit{start}$, such that $k$ is considered to be an usual high number of withdrawals for that time interval. A reference for the usual number of withdrawals on a certain time interval for a specific cardholder can be obtained from the gathered cardholder behavior (in our case represented as the \emph{withdrawal\_day} \texttt{Card} entity property of our defined data model).
Another indicator of this scenario to be considered could also be the \emph{amount} value of the withdrawal operations performed, which, in these scenarios, is normally a low value to prevent that the card owner realises.

\paragraph{III - Other Possible Fraud Scenarios\\\\}

Some other anomalous scenarios for which more graph patterns could be defined are:
\begin{itemize}
    \item \textbf{Anomalous location usage}: When a transaction is made in a location out of the threshold distance of the usual/registered address of the cardholder.
    \item \textbf{Anomalous number of operations}: Related with the II pattern characterized, we could also define a graph pattern related with a higher than average number of operations of any kind (withdrawal, inquiry, transfer or deposit) for a cardholder in a certain time interval.
    \item \textbf{Anomalous high expenses:} Similar to the II pattern, but in this case, not considering only the number of the withdrawal operations performed on a certain time interval, but the amount of the withdrawal operations on a certain time interval. This could indicate an anomalous behavior of the cardholder, withdrawing an amount of money way higher for a considered time interval.
\end{itemize}

%


\subsection{Data Model}

In the context of our work, the data we are considering can be seen as both \emph{static} and \emph{volatile} data, as we are considering a bank system application that, on the one hand, it contains all the \emph{static} information related to it about the cards, clients, accounts, among others, that a bank typically gathers. And, on the other hand, it is a system where transactions made by clients with their cards at ATMs, POS terminals, etc., are continuously occurring and received in a streaming manner: the \emph{volatile} data.\\

Therefore, due to this nature of the data, we consider a \emph{continuously evolving database} paradigm, where data can be both stable and volatile. Even though 
evolving databases can be implemented according to any approach, we decided to use the property graph data model (see \ref{prelim:graphdatamodel-pg}). Our property graph is a \emph{continuously evolving data graph}, which has a persistent (\emph{stable}) as well as non persistent (\emph{volatile}) relations. Stable relations correspond to edges occurring in standard graph databases while volatile relations are edges arriving in data streams during a set time interval.\\

The property graph data model provides us many benefits: (i) it is a simple and easy method to represent entities and their relationships in the form of graphs; (ii) it is a convenient model to represent dynamic data sources, in our case the continuously occurring relations between cards and ATMs; (iii) and it provides us a direct way to model queries related with fraud pattern matching. Finally, mention that we used \texttt{Neo4j} (see \ref{prelim:graphdbsystem-neo4j}) as the graph database management system to implement our property graph data model. \\

\subsubsection*{Design of the Property Graph Data Model}

In what follows we describe the design of our property graph taken data model. Due to the confidential and private nature of bank data, it was impossible to find a real bank dataset nor a real bank data model. In this regard, we developed our own proposal of a bank database model taking as standard reference the \emph{Wisabi Bank Dataset}, which is a fictional banking dataset publicly available in the Kaggle platform\cite{wisabi-bank-dataset}.\\

\begin{figure}[h]
  \centering
  \includegraphics[scale = 0.8]{images/1-DataModel/PG-behavior-complete.png}
  \caption{Complete property graph bank data model representation}
  \label{img:pg-complete}
\end{figure}

The proposed property graph data model is represented in Figure \ref{img:pg-complete}, consisting on both the stable and volatile property subgraphs merged.
The main difference and the primary reason for this separation is the semantics with which we intentionally define each of the subgraphs: the stable will be understood like a fixed static bank database, whereas the volatile will be understood as the data model to define the transactions, as continuous interactions between the entities of the model, which will not be permanently saved, but instead, only for a certain window of time under the mission of detecting anomalous bank operations. 
Note that we will only model the transaction interaction in the volatile subgraph, with no incidence in any other element of the architecture.
This separation will allow us to have a really simple and light property graph schema single-centered on the transactions with the minimal needed information 
(mostly identifiers of the entities and transaction links) and another, the stable, acting as a traditional bank database schema, from which to obtain the information details of the entities.
%%%

\paragraph*{Stable Property Graph\\\\}\label{section:stable-pg}

Taking into account the reference dataset model, we designed a simplified version, as shown in Figure \ref{img:pg-stable-def}, with the focus on representing and modeling card-ATM interactions. On it, the defined entities, relations and properties modeling the bank database are reduced to the essential ones, which are enough to create a relevant and representative bank data model, sufficient for the purposes of our work. Another option for the property graph data model representing a more common bank data model could be the one we defined in Figure \ref{img:pg-stable-big}, which intents to capture the data that a bank system database typically gathers. It consists of a more complete and possibly closer to reality data model, although unnecessarily complex for our objectives.\\

\begin{figure}[H]
  \centering
  \includegraphics[scale = 0.8]{images/1-DataModel/PG-stable-behavior-cards.png}
  \caption{\emph{Stable} property graph bank data model}
  \label{img:pg-stable-def}
\end{figure}

As a result, our definitive stable property graph model contains three node entities: \texttt{Bank}, \texttt{Card} and \texttt{ATM}, and three relations: \texttt{issued\_by} associating \texttt{Card} entities with the \texttt{Bank} entity, and \texttt{belongs\_to} and \texttt{interbank} associating the \texttt{ATM} entities with the \texttt{Bank} entity.\\

% Bank
\paragraph{\texttt{Bank}\\\\}

The \texttt{Bank} entity represents the bank we are considering in our system. Its properties consist
on the bank \emph{name}, its identifier \emph{code} and the location
of the bank headquarters, expressed in terms of \emph{latitude} and \emph{longitude}
coordinates, as seen in Table \ref{table:bank-node-properties}.\\
  
\begin{table}[H]
    \centering
      \begin{tabular}{|l|l|}
      \hline
      \textbf{Property}        & \textbf{Description}                                      \\ \hline
      \texttt{name}         & Bank name                                                 \\ \hline
      \texttt{code}         & Bank identifier code                                      \\ \hline
      \texttt{loc\_latitude}  & Bank headquarters GPS-location latitude                   \\ \hline
      \texttt{loc\_longitude} & Bank headquarters GPS-location longitude                  \\ \hline
      \end{tabular}
    \caption{Bank node properties}
    \label{table:bank-node-properties}
\end{table}

\paragraph{\texttt{ATM}\\\\}

The \texttt{ATM} entity represents the Automated Teller Machines (ATM) that either belong to the bank's network or that the bank can interact with.
% Potential possible generalization of the ATM entity to a POS entity
For the moment, this entity is understood as the classic ATM, however note that this entity could be potentially generalized to a \emph{Point-Of-Sale} (POS) entity, allowing a more general kind of interactions apart from the current Card-ATM interaction, where also POS terminal transactions could be included apart from the ATM ones. We distinguish two different kinds of ATMs, depending on their relation with the bank:
\begin{itemize}
  \item \textbf{Internal \texttt{ATMs}}: ATMs owned and operated by the bank. They are fully integrated within the
  bank's network. Modeled with the \texttt{belongs\_to} relation.
  \item \textbf{External \texttt{ATMs}}: These ATMs, while not owned by the bank, are still accessible for the bank
  customers to perform transactions. Modeled with the \texttt{interbank} relation. 
\end{itemize}

Both types of ATMs are considered to be of the same type of \texttt{ATM} node. Their difference is modeled as their relation with the bank instance: \texttt{belongs\_to} for the internal ATMs and \texttt{interbank} for the external ATMs.\\

\begin{table}[H]
    \centering
    \begin{tabular}{|l|l|}
    \hline
    \textbf{Name}        & \textbf{Description and value}                                      \\ \hline
    \texttt{ATM\_id}      & ATM unique identifier                             \\ \hline
    \texttt{loc\_latitude}  & ATM GPS-location latitude           \\ \hline
    \texttt{loc\_longitude} & ATM GPS-location longitude          \\ \hline
    \texttt{city}         & ATM city location                         \\ \hline
    \texttt{country}      & ATM country location                       \\ \hline
    \end{tabular}
    \caption{ATM node properties}
    \label{table:atm-node-properties}
\end{table}

The \texttt{ATM} node properties consist on the ATM unique identifier \emph{ATM\_id}, its location, expressed in terms of \emph{latitude} and \emph{longitude} coordinates, and the \emph{city} and 
\emph{country} in which it is located, as seen in Table \ref{table:atm-node-properties}.
Note that the last two properties are somehow redundant, considering that location coordinates
are already included. In any case both properties are maintained since their inclusion provides a more explicit and direct description of the location of the ATMs, which will be of special interest for some of the card anomalous patterns that will be considered.\\

\paragraph{\texttt{Card}\\\\}

The \texttt{Card} node represents the cards of the clients in the bank system. The \texttt{Card} node type properties, as depicted in Table
\ref{table:card-node-properties}, consist on the card unique 
identifier \emph{number\_id}, the associated client unique identifier \emph{client\_id}, the card validity expiration date \emph{expiration}, the Card Verification Code, \emph{CVC}, the coordinates of the associated client habitual residence address \emph{loc\_latitude} and 
\emph{loc\_longitude} and the \emph{extract\_limit} property, which represents the limit on the amount of money it can be extracted with the card on a single withdrawal, related with the the amount of money a person owns. These last two properties are of special interest for some future card fraud patterns to be considered. In the first case related with interactions far from the client's habitual residence address and in the second with unusually frequent or very high expenses interactions.\\

Finally, it contains the properties related with the \emph{behavior} of the client, representing the usual comportment of a client in regard with its ATM usage: \emph{amount\_avg\_withdrawal}, \emph{amount\_std\_withdrawal}, \emph{amount\_avg\_deposit}, 
\emph{amount\_std\_deposit}, \emph{amount\_avg\_transfer},\\  \emph{amount\_std\_transfer}, \emph{withdrawal\_day}, \emph{deposit\_day}, \emph{transfer\_day} and \emph{inquiry\_day}.
They are metrics representing the behavior of the owner of the Card, and they are included as properties as we think they could be of interest to allow the detection of some kinds of anomalies related with anomalous client's behavior in the future.

\begin{table}[H]
    \centering
    \begin{tabular}{|l|l|}
    \hline
    \textbf{Name}        & \textbf{Description and value}                                          \\ \hline
    \texttt{number\_id}   & Unique identifier of the card                                 \\ \hline
    \texttt{client\_id}   & Unique identifier of the client                               \\ \hline
    \texttt{expiration}   & Validity expiration date of the card                          \\ \hline
    \texttt{CVC}          & Card Verification Code                                        \\ \hline
    \texttt{extract\_limit} & Limit amount of money extraction associated with the card    \\ \hline
    \texttt{loc\_latitude}  & Client address GPS-location latitude                         \\ \hline
    \texttt{loc\_longitude} & Client address GPS-location longitude                        \\ \hline
    \end{tabular}
    \caption{Card node properties}
    \label{table:card-node-properties}
\end{table}

In the proposed property graph bank data model the client is completely anonymized in the system (no name, surname, age, or any other confidential details) by using only a \emph{client\_id}. Currently, \emph{client\_id} is included in the \texttt{Card} node type for completeness. However, it could be omitted for simplicity, as we assume a one-to-one relationship between a card and a client for the purposes of our work -- each card is uniquely associated with a single client, and each client holds only one card. Thus, the \emph{client\_id} is not essential at this stage but is retained in case the database model is expanded to support clients with multiple cards or cards shared among different clients.

\begin{figure}[H]
  \centering
  \includegraphics[scale = 0.7]{images/1-DataModel/PG-stable-edit-cardinal.png}
  \caption{Alternative - A more complex stable property graph bank data model. It consists of four node entities: \texttt{Bank}, \texttt{ATM}, \texttt{Client} and \texttt{Card} with their respective properties, and the corresponding relationships between them. The relations are: a directed relationship from \texttt{Client} to \texttt{Card} \texttt{owns} representing that a client can own multiple credit cards and that a card is owned by a unique client, then a bidirectional relation \texttt{has\_client} between \texttt{Client} and \texttt{Bank}; representing bank accounts of the clients in the bank. The relation between \texttt{Card} and \texttt{Bank} to represent that a card is \texttt{issued\_by} the bank, and that the bank can have multiple cards issued. Finally, the relations \texttt{belongs\_to} and \texttt{interbank} between the \texttt{ATM} and \texttt{Bank} entities, representing the two different kinds of ATMs depending on their relation with the bank; those ATMs owned and operated by the bank and those that, while not owned by the bank, are still accessible for the bank customers to perform transactions. This model allows a more elaborated representation of what a bank system database is. As it can be seen it represents clients as an independent entity from the \texttt{Card} entity, and it also allows to represent bank accounts through the relation between the \texttt{Client} and \texttt{Bank} entities. }
  \label{img:pg-stable-big}
\end{figure}


\paragraph*{Volatile Property Graph\\\\}\label{section:volatile-pg}

The volatile property graph consists on an abstraction of the property graph model to describe the interactions between the cards and the ATMs in the bank system. These interactions are going to be continuously occurring and arriving to our system as data stream.\\

The proposed property graph, represented on Figure \ref{img:pg-volatile}, is a subgraph of the original bank property graph model (Figure \ref{img:pg-complete}). It contains the \texttt{Card} and \texttt{ATM} entities with the minimal information needed to identify them -- \emph{number\_id} and \emph{ATM\_id}, Card and ATM identifiers, respectively -- between which the interaction occurs, along with additional details related to the interaction. Those identifiers are enough to be able to recover, if needed, the whole information about the specific \texttt{Card} or \texttt{ATM} entity in the stable property graph. In addition, it contains the \texttt{interaction} relationship between the \texttt{Card} and the \texttt{ATM} nodes. The \texttt{interaction} relation contains as properties (see table \ref{table:interaction-relation-properties}): \emph{id} as the interaction unique identifier, \emph{type} which describes the type of the interaction (withdrawal, deposit, balance inquiry or transfer), \emph{amount} describing the amount of money involved in the interaction in the local currency considered, and finally, \emph{start} and \emph{end} which define the interaction \emph{datetime} start and end moments, respectively. 

\begin{figure}[h]
    \centering
    \includegraphics[scale = 0.8]{images/1-DataModel/schema-volatile.png}
    \caption{Volatile Property Graph Data Model}
    \label{img:pg-volatile}
\end{figure}

\begin{table}[H]
    \centering
    \begin{tabular}{|l|l|}
    \hline
    \textbf{Property}        & \textbf{Description}                                      \\ \hline
    \texttt{id}      & Interaction/Transaction unique identifier                             \\ \hline
    \texttt{type}  & Transaction type: withdrawal, deposit, balance inquiry or transfer           \\ \hline
    \texttt{amount} & Money amount involved in the transaction          \\ \hline
    \texttt{start}         & Transaction start time moment                         \\ \hline
    \texttt{end}      & Transaction end time moment                     \\ \hline
    \end{tabular}
    \caption{Interaction relation properties}
    \label{table:interaction-relation-properties}
\end{table}

A key aspect that we consider in our data model is the division of the \texttt{interaction} relation in two edges -- the \emph{opening} and the \emph{closing} edges -- both forming a single \texttt{interaction} relation. The \emph{opening} edge (Figure \ref{img:opening-edge-1}) will be the indicator of the beginning of a new interaction between a Card and a ATM. It contains the values of the properties related with the starting time \emph{start}, the interaction \emph{type} as well as the \emph{id}. The \emph{closing} edge (Figure \ref{img:closing-edge-1}) will indicate the end of the interaction, completing the values of the rest of the properties of the interaction: \emph{end} and \emph{amount}.

\begin{figure}[H]
  \centering
  \includegraphics[scale = 0.8]{images/1-DataModel/2-edges-tx-tfm.png}
  \caption{\emph{Opening} interaction edge}
  \label{img:opening-edge-1}
\end{figure}

With this division of the \texttt{interaction} relation in two edges we are simulating that for each transaction, our system receives an initial message when the transaction starts and a final message once the transaction is finished on the ATM. This can be a key aspect in our system, since it can allow us to develop a system that is able not only to detect anomalous scenarios on interactions that have already been produced/closed, but also to act in real time before the anomalous transaction detected has actually occurred.

\begin{figure}[H]
  \centering
  \includegraphics[scale = 0.8]{images/1-DataModel/2-edges-tx-tfm-1.png}
  \caption{\emph{Closing} interaction edge}
  \label{img:closing-edge-1}
\end{figure}

\newpage
\section{The Query Model}

\textcolor{red}{TODO: Formal description of the qeury model\\}

\subsection{Fraud Pattern I - Card Cloning}

A first algorithmic proposal to detect this kind of fraud pattern is the one shown in the algorithm
\ref{alg:check-fraud-1}. Note that $S$ refers to the filter's subgraph and $e_{new}$ is the new incoming edge belonging to the filter, such that it is a opening interaction edge, since in the case it is a closing interaction edge, we do not perform the \text{CheckFraud()}.

\begin{algorithm}[H]
    \small
    \begin{algorithmic}[1]
    \REQUIRE $S$ is the subgraph of edges of the filter (sorted by time)
    \REQUIRE $e_{new}$ is the new incoming opening interaction edge belonging to the filter 
    \STATE $\texttt{fraudIndicator} \gets \texttt{False}$
    \STATE $i \gets |S|$
    \WHILE{$i > 0$ \AND \texttt{fraudIndicator} = \texttt{False}}
      \STATE $e_i \gets S[i]$
      \STATE $\texttt{t\_min} \gets \text{obtain\_t\_min}(e_i, e_{new})$
      \STATE $\texttt{t\_diff} \gets e_{new}.start - e_i.end$
      \IF{$\texttt{t\_diff} < \texttt{t\_min}$}   
        \STATE $\text{createAlert}(e_i, e_{new})$
        \STATE $\texttt{fraudIndicator} \gets \texttt{True}$
      \ENDIF
      \STATE $i \gets i - 1$
    \ENDWHILE
    \end{algorithmic}
    \caption{$\text{CheckFraud}(S, e_{new})$ -- \textbf{initial version}}
    \label{alg:check-fraud-1}
\end{algorithm}

There are some aspects and decisions of this algorithm that are worth to describe:

\begin{itemize}
    \item \textbf{Pairwise detection}. The checking of the anomalous fraud scenario is done doing the check between the new incoming edge $e_{new}$ and each of the edges $e_i$ of the filter's subgraph $S$.
    \item \textbf{Backwards order checking}. The pairs $(e_{new}, e_i)$ are checked in a backwards traversal order of the edge list of the subgraph $S$, starting with the most recent edge of the subgraph and ending with the oldest.  
    \item \textbf{Stop the checking whenever the first anomalous scenario is detected}. Whenever an anomalous scenario corresponding to a pair ($e_{new}, e_i)$, then we stop the checking at this point and emit the corresponding alert. Therefore we do not continue the checking with previous edges of $S$. 
    \item \textbf{Emission of the pair $(e_{new}, e_i)$ as the alert}. The alert is composed by the pair $(e_{new}, e_i)$ that is detected to cause the anomalous scenario. Both edges are emitted in the alert since we do not know which is the one that is the anomalous. On the one hand, it can be $e_i$, which is previous to $e_{new}$, in the case that $e_i$ at the moment it arrived it did not cause any alert with the previous edges/transactions of the subgraph and it causes it now with a new incoming edge $e_{new}$ which is a regular transaction of the client. On the other hand, it can be $e_{new}$, which is the last having arrived to the system, that it directly causes the alert with the last (ordinary) transaction of the card.
\end{itemize}

However, a more detailed study, lead us to a simplification of the initially proposed algorithm to the one shown in \ref{alg:check-fraud-def}. On it we just perform the checking between the new incoming edge $e_{new}$ and the most recent edge of the subgraph $S$, $e_{last}$.

\begin{algorithm}[H]
  \small
  \begin{algorithmic}[1]
  \REQUIRE $S$ is the subgraph of edges of the filter (sorted by time)
  \REQUIRE $e_{new}$ is the new incoming opening interaction edge belonging to the filter 
  \STATE $last \gets |S|$
  \STATE $e_{last} \gets S[last]$
  \STATE $\texttt{t\_min} \gets \text{obtain\_t\_min}(e_{last}, e_{new})$
  \STATE $\texttt{t\_diff} \gets e_{new}.start - e_{last}.end$
  \IF{$\texttt{t\_diff} < \texttt{t\_min}$}   
    \STATE $\text{createAlert}(e_{last}, e_{new})$
  \ENDIF
  \end{algorithmic}
  \caption{$\text{CheckFraud}(S, e_{new})$ -- \textbf{definitive version}}
  \label{alg:check-fraud-def}
\end{algorithm}


In what follows we argument the reason why it is sufficient to just check the fraud scenario among $e_{new}$ and the last/most recent edge of the subgraph and not have to continue having to traverse the full list of edges.

Assume that we have a subgraph as depicted in Figure \ref{img:fp-I-demo}, and that we do not know if there have been anomalous scenarios produced between previous pairs of edges of the subgraph. Name $F_I(y_i,y_j)$ a boolean function that is able to say whether it exists an anomalous fraud scenario of this type between the pair of edges $(y_i,y_j)$ or not. In addition, note that the edges of the subgraph $S$ are ordered by time in ascending order, in such a way that $y_1 < y_2 < y_3$. Finally note that $y_3 \equiv e_{new}$ as it is the new incoming edge and $y_2 \equiv e_{last}$, since it is the last edge / the most recent edge of $S$.

\begin{figure}[H]
  \centering
  \includegraphics[scale = 0.6]{images/2-QueryModel/fp-I-demo-1.png}
  \caption{Subgraph $S$ of a card -- Fraud Pattern I}
  \label{img:fp-I-demo}
\end{figure}

Note that we can have that:
\begin{itemize}
    \item $F_I(y_2,y_3)$: We emit an alert of this anomalous scenario produced between the pair $(y_2,y_3)$. We could continue checking for anomalous scenarios between $y_3$ and previous edges of the subgraph. However, what we consider important for the bank system is to detect the occurrence of an anomalous scenario in a certain card. Therefore, we consider that, to achieve this, it is enough to emit a single alert of anomalous scenario on this card, and not many related with the same incoming transaction of the same card.
    \item $\neg F_I(y_2,y_3)$: We analyze whether it would be interesting or not to continue the checking with previous edges of the subgraph, based on assumptions on the fraud checking between previous edges. In particular we can have two cases:
    \begin{itemize}
        \item If $F_I(y_1,y_2)$: Having this it can happen that either $F_I(y_1,y_3)$ or $\neg F_I(y_1,y_3)$. In the case of $F_I(y_1,y_3)$, since $\neg F_I(y_2,y_3)$, we can infer that the anomalous scenario detected between $y_1$ and $y_3$ is a continuation of the same previous anomalous scenario detected between $y_1$ and $y_2$. Therefore, we can conclude that this does not constitute a new anomalous scenario that would require an alert.
        \item If $\neg F_I(y_1,y_2)$: It can be shown that \emph{by transitivity}, having \\
        $\neg F_I(y_1,y_2) \land \neg F_I(y_2,y_3)
        \implies \neg F_I(y_1,y_3)$. \\
        \textcolor{red}{TODO: Show a formal demostration of this case!}
    \end{itemize}
\end{itemize}

Therefore, we have seen that, it is enough to perform the checking between the pair formed by $e_{new}$ and the most recent edge of the subgraph $e_{last}$. $\square$

\textcolor{red}{TODO: Explain that we use this proof as a way to show that we do not need to store the full list of edges in the case of this fraud pattern (just the last edge). Maybe for others we need to store more / a list of edges.}


\textcolor{red}{\rule{\textwidth}{1pt}}
\textcolor{red}{TODO: Complete other aspects of the filter worker algorithmic workflow\\}
Others -- not so much related with the CheckFraud algorithm, but in general with the filter's algorithm --:
\begin{itemize}
    \item Save all the edges in the subgraph $S$, even though they are the reason of the creation of an anomalous scenario.
    \item Number of anomalous fraud scenarios that can be detected. Bounded by:
    $$\#TX\_ANOM \leq SCENARIOS \leq 2*\#TX\_ANOM$$
    \textcolor{red}{TODO: Poner dibujo y explicar mejor}
\end{itemize}

\textcolor{red}{\rule{\textwidth}{1pt}}

\begin{graysection}

\begin{comment}
\begin{algorithm}[H]
  \begin{algorithmic}[1]
  \STATE $e_i \gets \texttt{edgeChannel}$ \COMMENT{$e_i \in \texttt{Filter}_i$}
  \IF{$e_i.\texttt{type} = \texttt{interaction-start}$}
      \FOR{$e_s$ in $\texttt{Subgraph}_i$}
          \IF{$e_s.\texttt{id} \ne e_i.\texttt{id}\ \land$ \\
          \hspace{3.1em} $e_i.\texttt{start} - e_s.\texttt{end}  < T_{min}(e_i.\texttt{loc}, e_s.\texttt{loc})$}
              \STATE $\texttt{emitAlert}(e_i, e_s)$
          \ENDIF   
      \ENDFOR
  \ENDIF
  \end{algorithmic}
  \caption{FP1 check algorithm pseudocode}
\end{algorithm}


\paragraph{Possible Improvement}

Note that the checking of the ${FP}_1$ with respect to the new incoming edge $e_i$ is
performed against all the previous stored edges $e_s$ in the subgraph.\\ 
$\rightarrow$ With some kind of \textit{marking} strategy of the ${FP}_1$ in between 
the previous edges of the subgraph, in some of the cases, possibly, it will not be needed
to perform the checking of ${FP}_1$ of $e_i$ with respect of all the previous added
edges $e_s$: ${FP}_1(e_i, e_s)$

\begin{equation}
  \begin{cases}
    \nexists {FP}_1(e_i, e_s) \\
    \nexists {FP}_1(e_{s}, e_{s-1})
  \end{cases}\implies \nexists {FP}_1(e_i, e_{s-1}), \hspace{2em} \forall s, 1 < s < i\
\end{equation}

In this case, if somehow, it is marked that $\nexists {FP}_1(e_{s}, e_{s-1})$, and now 
we check that $\nexists {FP}_1(e_i, e_s)$, then due to a possible transitivity property,
we can show that $\nexists {FP}_1(e_i, e_{s-1}) \hspace{1em} \forall s, 1 < s < i$. 

% Drawing

\textcolor{red}{\rule{\textwidth}{1pt}}

\end{comment}


\subsection{Initial Tests}

In the following we describe the preliminar tests that were done to check the correct
expected behavior of the ${FP}_1$ algorithm.

\subsubsection{${FP}_1$ - Test 1}

\begin{figure}[H]
  \hspace*{-2cm} 
  \includegraphics[scale=0.55]{images/2-QueryModel/FP1-test-1.png}
\end{figure}

\subsubsection{${FP}_1$ - Test 1.1}

\begin{figure}[H]
  \hspace*{-2cm} 
  \includegraphics[scale=0.45]{images/2-QueryModel/FP1-test-1.1.png}
\end{figure}

\end{graysection}
%\subsection{Architecture of the continuos query engine}


XXXXXX Gr\'afico de Javier
XXXXXx

\fmc{Explicar la idea de nuestro sistema, lo de que es un sistema para la detección de fraude a tiempo real, idea del double check para la authentication... sin entrar en mucho detalle de como lo hacemos (ya se describe en el apartado siguiente del continuous query engine). Que describa la idea / propósito / objetivo de lo que queremos hacer. Que sirva para introducir el apartado siguiente del \DPATM}

\fmc{IMPORTANTE: Buen dibujo de la arquitectura que sirva como para describir la idea del sistema que hacemos}
%
\iffalse
\begin{figure}[t!]
         \centering
         \includegraphics[width=0.7\textwidth]{images/architecture.png}
         \caption{Preliminary continuous query engine architecture for detecting anomalous ATM transactions based on the dynamic pipeline computational model. Considering the schema given in Figure \ref{fig:constinuousPGa}, in this directed (multi) graph presentation of the $\mathsf{DP_{ATM}}$, the arriving input data is a stream  $\mathsf{\langle \dots e_k\dots e_j \: e_{j-1}\rangle}$ corresponding to \textsf{interactions} (volatile relations). Boxes (vertices) represent stateful processes called \emph{stages} and internal arrows in the pipeline represent channels. Blue channels carry interaction edges and red channels carry detected anomalies (answers).  $\mathsf{S_r}$ and $\mathsf{S_k}$ correspond to the \textsf{Source} and \textsf{Sink} stages which receive input data and results, respectively. \textsf{Filter} stages, $\mathsf{F_{N}}$, are parameterized with the value of the property \textsf{number} ($\mathsf{N}$) of \textsf{Card} vertices. The \textsf{Generator} stage, $\mathsf{G_F}$, is in charge of spawning new filters, when required. The stable PG is a standard bank database (i.e. without volatile relations). Transactional log and Answers log keep input interactions and answers, respectively.
         }
         \label{fig:DP_ATM}
\end{figure}
\fi
%

%\input{sections/Proposal/3-DynamicPipeline/dp}
%
\begin{comment}
\begin{frame}{Continuous Query Engine: $\mathsf{DP_{ATM}}$}
    \begin{figure}
        \hspace*{-1cm} % adjust the value to control the leftward shift
        \includegraphics[width=1.2\textwidth]{figures/architecture.png}
    \end{figure}
\end{frame}
\end{comment}

\begin{comment}
\begin{frame}{$\mathsf{DP_{ATM}}$ - Input Stream}
\textbf{Note}: 2 edges per transaction - the \emph{opening} edge and the \emph{closing} edge.

\begin{figure}
    \centering
    \only<1>{\includegraphics[width=\textwidth]{images/1-DataModel/2-edges-tx-tfm.png}}
    \only<2>{\includegraphics[width=\textwidth]{images/1-DataModel/2-edges-tx-tfm-1.png}}
    \caption{\only<1>{Opening edge}\only<2>{Closing edge}}
\end{figure}
\end{frame}
\end{comment}

\begin{frame}{$\mathsf{DP_{ATM}}$ - Stages}
    \begin{itemize}
        \item<1-> \textbf{Source:} Manages the in-connection with the outside: streaming input / file reading... \& general transactions log.
        \item<2-> \textbf{Sink:} Manages the out-connection. Outside answers/alerts emission \& alert log.
        \item<3-> \textbf{Generator:} Generation of new filters.
        \item<4-> \textbf{Filter:} (Stateful) stage. Anomalous detection tracking process of \emph{maxFilterSize} cards.
    \end{itemize}
    \begin{figure}
    \hspace*{-0.9cm}
    \centering
    \only<1>{\includegraphics[scale=0.7]{images/Source.png}}
    \only<2>{\hspace{-0.1cm}\includegraphics[scale=0.7]{images/Sink.png}}
    \only<3>{\hspace{-0.1cm}\includegraphics[scale=0.7]{images/Generator.png}}
    \only<4>{\hspace{-0.2cm}\includegraphics[scale=0.7]{figures/Filter-edit.png}}
\end{figure}
\end{frame}


\begin{comment}
\begin{frame}{$\mathsf{DP_{ATM}}$ - Stages}
    \begin{itemize}
        \item<1-> \textbf{Source:} Manages the in-connection with the outside: streaming input / file reading... \& general transactions log.
        \item<2-> \textbf{Sink:} Manages the out-connection. Outside answers/alerts emission \& alert log.
        \item<3-> \textbf{Generator:} Generation of new filters.
        \item<4-> \textbf{Filter:} (Stateful) stage. Anomalous detection tracking process of \emph{maxFilterSize} cards.
    \end{itemize}
    
    \centering
    \vspace{-0.5cm}
    \begin{adjustwidth}{-0.7cm}{}
    \begin{tikzpicture}
        \begin{pgfonlayer}{nodelayer}
            \node [style=io, minimum height=1.25cm, minimum width=3.5cm, shape border rotate=270, visible on=<1->] (0) at (-4.5, 0) {Source};
            \node [style=filter_gen, minimum width=1cm, minimum height=2.5cm, visible on=<4->] (1) at (-2, 0) {Filter};
            \node [style=filter_gen, minimum width=1cm, minimum height=2.5cm, visible on=<4->] (2) at (0.4, 0) {Filter};
            \node [style=filter_gen, minimum width=1cm, minimum height=2.5cm, align=center, visible on=<3->] (3) at (3, 0) {Generator \\ \includegraphics[width=.05\textwidth]{gear} };
            \node [style=io, minimum height=1.25cm, minimum width=3.5cm, shape border rotate=90, visible on=<2->] (4) at (5.7, 0) {Sink};
        \end{pgfonlayer}
        \begin{pgfonlayer}{edgelayer}
            \draw [style={opChan}, visible on=<5->] ([yshift=0.5 cm]0.east) to["$e_i$"] ([yshift=0.5 cm]1.west);
            \draw [style={opChan}, visible on=<5->] ([yshift=0.5 cm]1.east) to["$e_i$"] ([yshift=0.5 cm]2.west);
            \draw [style={opChan}, visible on=<5->] ([yshift=0.5 cm]2.east) to["$e_i$"] ([yshift=0.5 cm]3.west);
            
            \draw [style={daChan}, visible on=<5->] ([yshift=-0.5 cm]1.east) to["Alerts"] ([yshift=-0.5 cm]2.west);
            \draw [style={daChan, visible on=<5->}] ([yshift=-0.5 cm]2.east) to["Alerts"] ([yshift=-0.5 cm]3.west);
            \draw [style={daChan}, visible on=<5->] ([yshift=-0.5 cm]3.east) to["Alerts"] ([yshift=-0.5 cm]4.west);
        \end{pgfonlayer}
        \node (down0) at (-4.5, -2.55);
        \draw [[-{Stealth[length=4mm]}, alt=<1>{red}{black}] (0) to["log"] (down0);
        
        \node (down4) at (5.7, -2.55);
        \draw [[-{Stealth[length=4mm]}, visible on=<2->, alt=<2>{red}{black}] (4) to["log" '] (down4);
        
        \node (right4) at (7.4, 0);
        \draw [[-{Stealth[length=4mm]}, visible on=<2->, red] (4) to["Alerts"] (right4);
    \end{tikzpicture}
    \end{adjustwidth}
\end{frame}
\end{comment}

\begin{frame}{$\mathsf{DP_{ATM}}$ - Filter Stage}
    \begin{columns} % Create two columns
        % Left column for the item list
        \begin{column}{0.5\textwidth}
            \begin{itemize}
                \item Stores the \textbf{induced card subgraph(s)} of the incoming belonging edges.
                \item Evaluation of continuous query pattern(s). Each has a \textbf{connection} session with the \textbf{stable gdb}.
                \item Emission of \textbf{alerts} in case of matching a query pattern.
                \item Filter \emph{worker} $\mathsf{FW}$ to avoid bottlenecks - \emph{in parallel}.
            \end{itemize}
        \end{column}

        % Right column for the image
        \begin{column}{0.6\textwidth}
            \centering
            \includegraphics[scale=0.40]{images/3-Engine/filter-worker-subgraphs.png}
        \end{column}
    \end{columns}
\end{frame}

\begin{comment}
\begin{frame}[fragile]{$\mathsf{DP_{ATM}}$ - Filter Stage}
\begin{itemize}
    \item Card cloning fraud pattern algorithmic evaluation
\end{itemize}
    \scalebox{0.9}{ 
    \hspace{-0.7cm}
    \begin{minipage}{1.2\textwidth} 
    \begin{algorithm}[H]
        \caption{\texttt{checkFraud}($\mathsf{S_c, e_{new}}$)}
        \label{alg:check-fraud-impl}
        \begin{algorithmic}[1]
            \Require $\mathsf{S_c}$ is a non-empty subgraph of interaction edges of card $\mathsf{c}$, 
            $\mathsf{e_{new}}$ is the \texttt{Edge} related to the new incoming opening interaction \texttt{EdgeStart} of card $\mathsf{c}$
            \State $\mathsf{e_{last}} \gets S_c[|S_c| - 1]$ \Comment{Retrieve last edge from $\mathsf{S_c}$}
            \If{$\mathsf{e_{last}}.\texttt{Tx\_end}$ is empty}
                \State \texttt{LOG: Warning: A tx ($\mathsf{e_{new}}$) starts before the previous ($\mathsf{e_{last}}$) ends!}
                \Return
            \EndIf
            \If{$\mathsf{e_{last}}.\texttt{ATM\_id} \neq \mathsf{e_{new}}.\texttt{ATM\_id}$}
                \State $\texttt{t\_min} \gets \text{obtainTmin}(\mathsf{e_{last}}, \mathsf{e_{new}})$
                \State $\texttt{t\_diff} \gets \mathsf{e_{new}}.\texttt{Tx\_start} - \mathsf{e_{last}}.\texttt{Tx\_end}$
                \If{$\texttt{t\_diff} < \texttt{t\_min}$}   
                    \State $\text{emitAlert}(\mathsf{e_{last}}, \mathsf{e_{new}})$
                \EndIf
            \EndIf
        \end{algorithmic}
    \end{algorithm}
    \end{minipage}
    }
\end{frame}
\end{comment}

\begin{comment}
\begin{frame}{$\mathsf{DP_{ATM}}$ - Filter Stage}
    \begin{itemize}
         \item \textbf{Filter \emph{worker}}.
         \item Simultaneously (many possible) different continuous query patterns: $FP_1, FP_2, ..., FP_n$.
    \end{itemize}

    \begin{figure}
        \centering
        \includegraphics[scale=0.32]{figures/engine-filter-worker-1.png}
   \end{figure}
\end{frame}
\end{comment}

\begin{frame}{Architecture}
\begin{figure}
    \centering
    \hspace*{-1cm}
    \includegraphics[width=1.18\linewidth]{images/architectureCQE.png}
\end{figure}
\end{frame}

